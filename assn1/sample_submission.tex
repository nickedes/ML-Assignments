\documentclass[a4paper,11pt]{article}

\usepackage{mlsubmit}

\begin{document}

\initmlsubmision{1}                              					% assignment number
								{Nikhil Mittal}      						           		% your name
								{17111056}																		% your roll number

\begin{mlsolution}

Lerning with prototypes:\\
A two class problem where the prototypes are the points (1, 0) (green) and (0, 1) (red). Let the green point be \begin{math} \mu _{+} (1, 0) \end{math} and red point be \begin{math} \mu _{-} (0, 1) \end{math}. The decision boundary is the hyperplane where any point lying on the hyperplane is at equal distance from both the prototypes.
\\
1.  \begin{math}
 d\left ( z^{1}, \space z^{2} \right ) = \left \langle z^{1} - z^{2}, \space U\left ( z^{1} - z^{2} \right ) \right \rangle, U = \begin{bmatrix} 
3 & 0 \\ 0 & 1 \end{bmatrix} \\ 
\end{math}
\\
Let's take a point z \begin{math} \in \mathbb{R}^{2} \end{math}, finding it's distance from both the prototypes :
\\
So, z = \begin{math}
\begin{bmatrix} x\\ y \end{bmatrix},
\end{math} 
\begin{math}
\mu _{+} = \begin{bmatrix} 1\\ 0 \end{bmatrix} 
\end{math}
and
\begin{math}
\mu _{-} = \begin{bmatrix} 0\\ 1 \end{bmatrix}
\end{math}
\newline
\newline
\begin{math}
d\left ( \mu_{+}, \space z \right ) = \left \langle \mu_{+} - z, \space U\left ( \mu_{+} - z \right ) \right \rangle
\end{math}
\newline
\newline
\begin{math}
d\left ( \mu_{+}, \space z \right ) = \left \langle \begin{bmatrix} 1\\ 0 \end{bmatrix} - \begin{bmatrix} x\\ y \end{bmatrix} , \begin{bmatrix} 
3 & 0 \\ 0 & 1 \end{bmatrix} \left (  \begin{bmatrix} 1\\ 0 \end{bmatrix} - \begin{bmatrix} x\\ y \end{bmatrix}  \right ) \right \rangle
\end{math}
\newline
\begin{math}
d\left ( \mu_{+}, \space z \right ) = \left \langle \begin{bmatrix} 1-x\\ -y \end{bmatrix} , \begin{bmatrix} 
3 & 0 \\ 0 & 1 \end{bmatrix} \left (  \begin{bmatrix} 1-x\\ -y \end{bmatrix}  \right ) \right \rangle
\end{math}
\newline
\begin{math}
d\left ( \mu_{+}, \space z \right ) = \left \langle \begin{bmatrix} 1-x\\ -y \end{bmatrix} , \begin{bmatrix} 3*(1-x) \\ -y \end{bmatrix} \right \rangle
\\
\end{math}
\begin{math}
d\left ( \mu_{+}, \space z \right ) = \begin{bmatrix} 1-x\\ -y \end{bmatrix} ^{T}.\begin{bmatrix} 3*(1-x) \\ -y \end{bmatrix}\\ \\
d\left ( \mu_{+}, \space z \right ) = 3(1-x)^{2} + y^{2}
\end{math}
\newline
\begin{math}
\\
d\left ( \mu_{-}, \space z \right ) = \left \langle \mu_{-} - z, \space U\left ( \mu_{-} - z \right ) \right \rangle
\end{math}
\newline
\newline
\begin{math}
d\left ( \mu_{-}, \space z \right ) = \left \langle \begin{bmatrix} 0\\ 1 \end{bmatrix} - \begin{bmatrix} x\\ y \end{bmatrix} , \begin{bmatrix} 
3 & 0 \\ 0 & 1 \end{bmatrix} \left (  \begin{bmatrix} 0\\ 1 \end{bmatrix} - \begin{bmatrix} x\\ y \end{bmatrix}  \right ) \right \rangle
\end{math}
\newline
\begin{math}
d\left ( \mu_{-}, \space z \right ) = \left \langle \begin{bmatrix} -x\\ 1-y \end{bmatrix} , \begin{bmatrix} 
3 & 0 \\ 0 & 1 \end{bmatrix} \left (  \begin{bmatrix} -x\\1 -y \end{bmatrix}  \right ) \right \rangle
\end{math}
\newline
\begin{math}
d\left ( \mu_{-}, \space z \right ) = \left \langle \begin{bmatrix} -x\\ 1-y \end{bmatrix} , \begin{bmatrix} -3x \\ 1-y \end{bmatrix} \right \rangle
\end{math}
\newline
\begin{math}
\\
d\left ( \mu_{-}, \space z \right ) = \begin{bmatrix} -x\\ 1-y \end{bmatrix} ^{T}.\begin{bmatrix} -3x \\ 1-y \end{bmatrix}\\ \\
d\left ( \mu_{-}, \space z \right ) = 3x^{2} + (1-y)^{2}\\
\end{math}
Equate the difference of distances to decision boundary,\\
\begin{math}
f(z) = d\left ( \mu_{+}, \space z \right ) - d\left ( \mu_{-}, \space z \right )\\
\end{math}
For equation of f(z),
\begin{math}
f(z) = 0 \\
Then, d\left ( \mu_{+}, \space z \right ) - d\left ( \mu_{-}, \space z \right ) = 0\\
3(1-x)^{2} + y^{2} - 3x^{2} - (1-y)^{2} = 0\\
3(1 + x^{2} - 2
x) + y^{2} - 3x^{2} - (1 + y^{2} - 2y) = 0\
3 + 3x^{2} - 6x + y^{2} - 3x^{2} - 1 - y^{2} - 2y = 0\\
-6x + 2y + 2 = 0\\
y = 3x - 1
\end{math}\\\\
The mathematical expression for the decision boundary is :\\
 \begin{center}\boxed{ y = 3x - 1}\end{center}
    
\begin{figure}[th]%
\centering
\includegraphics[width=0.3\columnwidth]{proto_blank.png}%
\hfill
\includegraphics[width=0.5\columnwidth]{Q1-1.png}%
\caption{Learning with Prototypes: the figure on the left shows the two prototypes. The figure on the right shows what the decision boundary if the distance measure used is $d(\vz^1,\vz^2) = \left \langle \vz^1-\vz^2, V (\vz^1-\vz^2) \right \rangle$, for any two points $\vz^1,\vz^2 \in \bR^2$. The decision boundary in this case is the line $y = 3x-1$.}%
\label{fig:proto1}%
\end{figure}

2.  \begin{math}
 d\left ( z^{1}, \space z^{2} \right ) = \left \langle z^{1} - z^{2}, \space V \left ( z^{1} - z^{2} \right ) \right \rangle, V = \begin{bmatrix} 
1 & 0 \\ 0 & 0 \end{bmatrix} \\ \end{math}

Let's take a point z \begin{math} \in \mathbb{R}^{2} \end{math}, finding it's distance from both the prototypes :
\\
So, z = \begin{math}
\begin{bmatrix} x\\ y \end{bmatrix},
\end{math} 
\begin{math}
\mu _{+} = \begin{bmatrix} 1\\ 0 \end{bmatrix} 
\end{math}
and
\begin{math}
\mu _{-} = \begin{bmatrix} 0\\ 1 \end{bmatrix}
\end{math}
\newline
\newline
\begin{math}
d\left ( \mu_{+}, \space z \right ) = \left \langle \mu_{+} - z, \space V\left ( \mu_{+} - z \right ) \right \rangle
\end{math}
\newline
\newline
\begin{math}
d\left ( \mu_{+}, \space z \right ) = \left \langle \begin{bmatrix} 1\\ 0 \end{bmatrix} - \begin{bmatrix} x\\ y \end{bmatrix} , \begin{bmatrix} 
1 & 0 \\ 0 & 0 \end{bmatrix} \left (  \begin{bmatrix} 1\\ 0 \end{bmatrix} - \begin{bmatrix} x\\ y \end{bmatrix}  \right ) \right \rangle
\end{math}
\newline
\begin{math}
d\left ( \mu_{+}, \space z \right ) = \left \langle \begin{bmatrix} 1-x\\ -y \end{bmatrix} , \begin{bmatrix} 
1 & 0 \\ 0 & 0 \end{bmatrix} \left (  \begin{bmatrix} 1-x\\ -y \end{bmatrix}  \right ) \right \rangle
\end{math}
\newline
\begin{math}
d\left ( \mu_{+}, \space z \right ) = \left \langle \begin{bmatrix} 1-x\\ -y \end{bmatrix} , \begin{bmatrix} 1-x \\ 0 \end{bmatrix} \right \rangle
\\
\end{math}
\begin{math}
d\left ( \mu_{+}, \space z \right ) = \begin{bmatrix} 1-x\\ -y \end{bmatrix} ^{T}.\begin{bmatrix} 1-x \\ -y \end{bmatrix}\\ \\
d\left ( \mu_{+}, \space z \right ) = (1-x)^{2}
\end{math}
\newline
\begin{math}
\\
d\left ( \mu_{-}, \space z \right ) = \left \langle \mu_{-} - z, \space V\left ( \mu_{-} - z \right ) \right \rangle
\end{math}
\newline
\newline
\begin{math}
d\left ( \mu_{-}, \space z \right ) = \left \langle \begin{bmatrix} 0\\ 1 \end{bmatrix} - \begin{bmatrix} x\\ y \end{bmatrix} , \begin{bmatrix} 
1 & 0 \\ 0 & 0 \end{bmatrix} \left (  \begin{bmatrix} 0\\ 1 \end{bmatrix} - \begin{bmatrix} x\\ y \end{bmatrix}  \right ) \right \rangle
\end{math}
\newline
\begin{math}
d\left ( \mu_{-}, \space z \right ) = \left \langle \begin{bmatrix} -x\\ 1-y \end{bmatrix} , \begin{bmatrix} 
1 & 0 \\ 0 & 0 \end{bmatrix} \left (  \begin{bmatrix} -x\\1 -y \end{bmatrix}  \right ) \right \rangle
\end{math}
\newline
\begin{math}
d\left ( \mu_{-}, \space z \right ) = \left \langle \begin{bmatrix} -x\\ 1-y \end{bmatrix} , \begin{bmatrix} -x \\ 0 \end{bmatrix} \right \rangle
\end{math}
\newline
\begin{math}
\\
d\left ( \mu_{-}, \space z \right ) = \begin{bmatrix} -x\\ 1-y \end{bmatrix} ^{T}.\begin{bmatrix} -x \\ 0 \end{bmatrix}\\ \\
d\left ( \mu_{-}, \space z \right ) = x^{2} \\
\end{math}
Equate the difference of distances to decision boundary,\\
\begin{math}
f(z) = d\left ( \mu_{+}, \space z \right ) - d\left ( \mu_{-}, \space z \right )\\
\end{math}
For equation of f(z),
\begin{math}
f(z) = 0 \\
Then, d\left ( \mu_{+}, \space z \right ) - d\left ( \mu_{-}, \space z \right ) = 0\\
(1-x)^{2} - x^{2} = 0\\
1 + x^{2} - 2x - x^{2} = 0\\
1 - 2x = 0\\
x = \frac{1}{2} \\
\end{math}\\
The mathematical expression for the decision boundary is :\\
\begin{center} \begin{math} x = \frac{1}{2} \end{math} \end{center}

\begin{figure}[th]%
\centering
\includegraphics[width=0.3\columnwidth]{proto_blank.png}%
\hfill
\includegraphics[width=0.6\columnwidth]{Q1-2.png}%
\caption{Learning with Protypes: the figure on the left shows the two prototypes. The figure on the right shows what the decision boundary if the distance measure used is $d(\vz^1,\vz^2) =\left \langle \vz^1-\vz^2, U (\vz^1-\vz^2) \right \rangle$, for any two points $\vz^1,\vz^2 \in \bR^2$. The decision boundary in this case is the line $x = 1/2$.}%
\label{fig:proto2}%
\end{figure}

\end{mlsolution}

\begin{mlsolution}


Lorem ipsum dolor sit amet, consectetur adipiscing elit. Duis feugiat vehicula dolor, sed ultricies leo. Phasellus euismod dictum felis in euismod. Proin pretium vel neque in placerat. Proin imperdiet egestas vulputate. Etiam faucibus accumsan ante non viverra. Duis ultrices ac odio vel sodales. In maximus gravida dolor, ut commodo lacus. Pellentesque ante massa, venenatis id aliquam et, posuere sed dui. Duis dignissim justo sit amet augue posuere fringilla. Suspendisse at nisi gravida, mattis justo sit amet, elementum elit. Praesent et massa ornare, consequat dui eget, ornare risus. Duis est nibh, sollicitudin nec mattis non, mattis in leo. Donec finibus justo sed massa sagittis, non fermentum nibh dictum. Pellentesque et congue purus. Donec porta pretium porttitor.

Morbi euismod risus eu tortor ornare malesuada. Nunc sed sollicitudin neque, efficitur rhoncus tellus. Cras malesuada augue arcu. Sed sem odio, tincidunt quis laoreet ac, facilisis ut nibh. Quisque gravida dolor at egestas aliquam. Aenean mollis massa sit amet enim mattis, vel fermentum tortor facilisis. Donec pellentesque est velit, vitae posuere lorem tristique ut.

Fusce pulvinar convallis lobortis. Mauris iaculis lacus vitae dui suscipit, ut ornare neque placerat. In mattis malesuada rutrum. Vivamus consectetur tempus ex sit amet aliquam. In blandit libero at mi rutrum, nec iaculis orci cursus. Maecenas a dolor lorem. Donec pretium turpis sapien, dapibus sollicitudin odio scelerisque eget. Maecenas egestas tellus a quam scelerisque, et pretium magna condimentum. In dapibus feugiat ornare. Nunc eget nulla convallis, laoreet tortor nec, convallis dui. Etiam in leo vitae nulla facilisis congue. Curabitur blandit sodales augue. Vivamus et aliquam orci, non suscipit elit. Quisque vestibulum lacus at velit congue semper.

Nulla efficitur risus nunc, in posuere turpis tempor eget. Sed efficitur id tellus non vestibulum. Praesent elementum condimentum sollicitudin. Integer eget quam dictum, varius est sit amet, aliquam mauris. Vestibulum ante ipsum primis in faucibus orci luctus et ultrices posuere cubilia Curae; Morbi in pretium dui. Sed luctus magna rutrum ex mollis, ut blandit lacus tincidunt. In tincidunt urna neque, placerat consequat velit sagittis id. Morbi pretium maximus fermentum. Interdum et malesuada fames ac ante ipsum primis in faucibus. Duis vehicula efficitur rhoncus. Nullam lacinia semper scelerisque. Maecenas eleifend nisi et ante auctor, a tincidunt arcu molestie. Aenean faucibus feugiat arcu ac mattis.
\begin{figure}[th]%
\centering
\includegraphics[width=0.5\columnwidth]{proto_blank.png}%
\hfill
\includegraphics[width=0.5\columnwidth]{proto_euclid_sample.png}%
\caption{Learning with Prototypes: the figure on the left shows the two prototypes. The figure on the right shows what the decision boundary if the distance measure used is $d(\vz^1,\vz^2) = \norm{\vz^1-\vz^2}_2$, for any two points $\vz^1,\vz^2 \in \bR^2$. The decision boundary in this case is the line $y = x$.}%
\label{fig:proto}%
\end{figure}
Morbi quis suscipit sapien. Pellentesque pulvinar fermentum tellus at malesuada. Nam id metus vitae risus dignissim laoreet. Pellentesque massa velit, vehicula in convallis et, vestibulum sed turpis. In venenatis massa vel mattis tincidunt. Donec varius faucibus elit, in blandit metus interdum sit amet. Nullam vel nibh non nisl mollis volutpat. Donec cursus iaculis lorem, id elementum metus iaculis nec. Cras a diam porttitor, suscipit dolor id, vestibulum nisi. Morbi maximus mauris a iaculis hendrerit. Duis rutrum quam in ex lobortis gravida. Aenean iaculis lacinia metus. Fusce sit amet dignissim elit, sed sodales lacus. Mauris ac neque finibus, bibendum tortor id, scelerisque neque. In nec quam ullamcorper, egestas sem ut, iaculis ante. Nam porta diam ut lacus sagittis euismod. 
\end{mlsolution}

\begin{mlsolution}
We need to design a likelihood and prior distribution such that \begin{math}
\stackrel\frown{w}_{fr} \end{math} this is the map estimate.
\\The Posterior distribution on w can be seen as proportional to the product of the likelihood and the prior distributions.

Let's take the gaussian likelihood and gaussian prior :
\\\\
\begin{math}
P[y^{i}|x^{i}, w] = \mathcal{N} \left ( \left \langle w, x^{i} \right \rangle, \sigma^{2} \right ) 
\end{math}\\ \\
\begin{math}
= \frac{1}{{\sigma \sqrt {2\pi } }}e^{{{ - \left( {y^{i} - \left \langle w, x^{i} \right \rangle } \right)^2 } \mathord{\left/ {\vphantom {{ - \left( {y^{i} - \left \langle w, x^{i} \right \rangle } \right)^2 } {2\sigma ^2 }}} \right. \kern-\nulldelimiterspace} {2\sigma ^2 }}}
\end{math}
\\\\So Log likelihood will be :
\\\\
\begin{math}
log P[\textbf{y} | \textbf{X}, \textbf{w}] = C - \frac{1}{2\sigma^{2}} \sum_{i = 1}^{n} \left ( y^{i} - \left \langle \textbf{w}, \textbf{x}^{i} \right \rangle \right )^{2}
\end{math}\\\\Here \textbf
X
 is \textbf{n}*\textbf{d} matrix where \textbf{n} is the no. of data points and \textbf{d} is the no. of features. \\\\Taking {}Gaussian prior, with a minor modification of assuming a \begin{math} \beta \end{math} vector whose each component \begin{math} \beta_{j} \end{math} is multiplied with each of the component of the \begin{math}\textbf{w}^{T}.\textbf{w}\end{math}
\\
Such that \begin{math} \beta = \left \{ \beta_{1}, \beta_{2}, ..., \beta_{d} \right \} , \beta_{j} > 0\end{math}  and d is the dimension.

So, 
\begin{math}
P[\textbf{w}] = \mathcal{N}\left ( 0, \rho^{2}.I_{d} \right ) = \frac{1}{\sqrt{\left ( 2\pi \right)^{d}\rho^{2}}}\exp\left ( - \frac{\sum_{j = 1}^{d} \beta_{j}\left ( \textbf{w}_{j} \right )^{2} }{2\rho^{2}} \right )
\end{math}
\\Taking log we get :\\

\begin{math}
\log \end{math}\;P\begin{math}[\textbf{w}]  = C^{'} - \frac{1}{2\rho^{2}} \sum_{j = 1}^{d} \beta_{j}\left ( \textbf{w}_{j} \right )^{2} 
\end{math}\\\\So the posterior distribution on \textbf{w} :\\

P \begin{math}[\textbf{w}\mid\textbf{X},\textbf{y}]  \propto \end{math}\;P\begin{math}[\textbf{y}|\textbf{X}, \textbf{w}] . \end{math}\;P\begin{math}[\textbf{w}]\end{math}\\

\begin{math}
\log P[\textbf{w}\mid\textbf{X},\textbf{y}]  = \log P[\textbf{y}|\textbf{X}, \textbf{w}] + \log P[\textbf{w}]
\end{math}\\

\begin{math}
\widehat{\textbf{w}}_{\textbf{MAP}}  = \arg \underset{\textbf{w}}{\max} \log P[\textbf{y}|\textbf{X}, \textbf{w}] + \log P[\textbf{w}]
\end{math}\\
Using the expressions of log-likelihood and log-prior, we can say that :\\

\begin{math}
\widehat{\textbf{w}}_{\textbf{MAP}}  = \arg \underset{\textbf{w}}{\max} \;
C - \frac{1}{2\sigma^{2}} \sum_{i = 1}^{n} \left ( y^{i} - \left \langle \textbf{w}, \textbf{x}^{i} \right \rangle \right )^{2} + C^{'} - \frac{1}{2\rho^{2}} \sum_{j = 1}^{d} \beta_{j}\left ( \textbf{w}_{j} \right )^{2}
\end{math}\\\\
Ignoring constants \\

\begin{math}
\widehat{\textbf{w}}_{\textbf{MAP}}  = \arg \underset{\textbf{w}}{\min} \;
\frac{1}{2\sigma^{2}} \sum_{i = 1}^{n} \left ( y^{i} - \left \langle \textbf{w}, \textbf{x}^{i} \right \rangle \right )^{2} + \frac{1}{2\rho^{2}} \sum_{j = 1}^{d} \beta_{j}\left ( \textbf{w}_{j} \right )^{2}
\end{math}\\\\
Multiplying the equation by \begin{math}2\sigma^{2}\end{math}, we get Equation 1:\\

\begin{math}
\widehat{\textbf{w}}_{\textbf{MAP}}  = \arg \underset{\textbf{w}}{\min} \; \sum_{i = 1}^{n} \left ( y^{i} - \left \langle \textbf{w}, \textbf{x}^{i} \right \rangle \right )^{2} + \frac{\sigma^{2}}{\rho^{2}} \sum_{j = 1}^{d} \beta_{j}\left ( \textbf{w}_{j} \right )^{2}
\end{math}\\\\
Let's take \begin{math}\alpha_{j}\end{math} , such that \begin{math} \alpha_{j} = \frac{\sigma^{2}}{\rho^{2}}.\beta_{j}  \end{math}.
Such that \begin{math} \alpha = \left \{ \alpha_{1}, \alpha_{2}, ..., \alpha_{d} \right \} , \alpha_{j} > 0\end{math}  and d is the dimension.
So Equation 1 changes to :\\

\begin{math}
\widehat{\textbf{w}}_{\textbf{MAP}}  = \arg \underset{\textbf{w}}{\min} \; \sum_{i = 1}^{n} \left ( y^{i} - \left \langle \textbf{w}, \textbf{x}^{i} \right \rangle \right )^{2} + \sum_{j = 1}^{d} \alpha_{j}\left ( \textbf{w}_{j} \right )^{2}
\end{math}\\\\This is similar to the \begin{math}\widehat{\textbf{w}}_{fr}\end{math} as given in the question.\\It has a closed form solution which can be derived as follows : \\\\Take derivative of \begin{math}\widehat{\textbf{w}}_{fr}\end{math} w.r.t \textbf{w}\\\\

\begin{math}
\frac{ d\widehat{\textbf{w}}_{\textbf{fr}} }{dx} = \left ( 2 \right ) \sum_{i = 1}^{n}\left ( y^{i} - \left \langle \textbf{w}, \textbf{x}^{i} \right \rangle \right )\left ( - \textbf{x}^{i} \right ) + \frac{d}{dw} (\sum_{j = 1}^{d} \alpha_{j}\left ( \textbf{w}_{j} \right )^{2})
\end{math}\\

\begin{math}
\frac{ d\widehat{\textbf{w}}_{\textbf{fr}} }{dx} = \left ( -2 \right ) \sum_{i = 1}^{n}\left ( \textbf{x}^{i}y^{i} - \textbf{x}^{i}\left \langle \textbf{w}, \textbf{x}^{i} \right \rangle \right  ) + \sum_{j = 1}^{d} \frac{d}{d\textbf{w}_{j}} (\alpha_{j}\left ( \textbf{w}_{j} \right )^{2})
\end{math}\\

\begin{math}
\frac{ d\widehat{\textbf{w}}_{\textbf{fr}} }{dx} = \left ( -2 \right ) \sum_{i = 1}^{n}\left ( \textbf{x}^{i}y^{i} - \textbf{w}. (\textbf{x}^{i})^{2} \right  ) + \sum_{j = 1}^{d} (2\alpha_{j}\textbf{w}_{j})
\end{math}\\

\begin{math}
\frac{ d\widehat{\textbf{w}}_{\textbf{fr}} }{dx} = \left ( -2 \right ) \sum_{i = 1}^{n}\left ( \textbf{x}^{i}y^{i} \right  )  + 2 \; \textbf{w}.\sum_{i=1}^{n} (\textbf{x}^{i})^{2} + 2 \alpha^{T}\textbf{w}
\end{math}\\\\ Equating this equation to zero to get the value of \textbf{w}, We get :\\

\begin{math}
\textbf{w}.\sum_{i=1}^{n} (\textbf{x}^{i})^{2} + \alpha^{T}\textbf{w} = \sum_{i = 1}^{n}\left ( \textbf{x}^{i}y^{i} \right  ) 
\end{math}\\

\begin{math}
\textbf{w}\textbf{X}^{T}\textbf{X} + \alpha^{T}\textbf{w} = \textbf{X}^{T}\textbf{Y}
\end{math}\\

\begin{math}
\textbf{w}\left ( \textbf{X}^{T}\textbf{X} + \alpha^{T}I \right ) = \textbf{X}^{T}\textbf{Y}\end{math}\\

\begin{math}
\textbf{w} = \left ( \textbf{X}^{T}\textbf{X} + \alpha^{T}I \right )^{-1} \textbf{X}^{T}\textbf{Y}
\end{math}\\
\\So this is the closed form expression for \begin{math}\widehat{\textbf{w}}_{fr}\end{math} :

\begin{math}
\widehat{\textbf{w}}_{fr} = \left ( \textbf{X}^{T}\textbf{X} + \alpha^{T}I \right )^{-1} \textbf{X}^{T}\textbf{Y}
\end{math}\\

\end{mlsolution}

\begin{mlsolution}

\begin{math}
\left \{ \widehat{\textbf{W}}, \left \{ \widehat{\xi _{i}} \right \} \right \} = \underset{\textbf{W},}{arg} \; \underset{ \left \{ \widehat{\xi _{i}} \right \}}{min} \sum_{k=1}^{K} \left \| \textbf{w}^{k} \right \|^{2}_{2} + \sum_{i=1}^{n} \xi _{i}\end{math}\\\\

\begin{math}
\;\;\;\;s.t. \left \langle \textbf{w}^{y^{i}}, x^{i} \right \rangle \geq \left \langle \textbf{w}^{k}, x^{i} \right \rangle + 1 - \xi_{i}\; , \forall i\; \forall k\neq y^{i} \\
\; \; \; \; \xi_{i} \geq 0, \forall i
\end{math}

So \begin{math}\widehat{\xi _{i}}\end{math} is for every data point.

The given constraint: \; \; \begin{math} \left \langle \textbf{w}^{y^{i}}, x^{i} \right \rangle \geq \left \langle \textbf{w}^{k}, x^{i} \right \rangle + 1 - \xi_{i}\; , \forall i\; \forall k\neq y^{i} \\ \end{math} \\

\begin{math} \xi_{i} \geq \left \langle \textbf{w}^{k}, x^{i} \right \rangle + 1 - \left \langle \textbf{w}^{y^{i}}, x^{i} \right \rangle \; , \forall i\; \forall k\neq y^{i} \\ \end{math} \\

\begin{math}
\xi_{i} \geq 1 + \left \langle \textbf{w}^{k}, x^{i} \right \rangle - \left \langle \textbf{w}^{y^{i}}, x^{i} \right \rangle \; , \forall i\; \forall k\neq y^{i} \\
\end{math}

So we can safely say that :
\begin{math}
\xi_{i} = \max\left (   1 + \left \langle \textbf{w}^{k}, x^{i} \right \rangle - \left \langle \textbf{w}^{y^{i}}, x^{i} \right \rangle \; \right ), \forall i,\; \forall k\neq y^{i} \\
\end{math}

As,\begin{math} \eta ^{i} = \left \langle \textbf{W}, \textbf{x}^{i} \right \rangle \Rightarrow \eta_{k} ^{i} = \left \langle \textbf{w}^{k}, \textbf{x}^{i} \right \rangle  and \;\eta_{y} ^{i} = \left \langle \textbf{w}^{y^{i}}, \textbf{x}^{i} \right \rangle \end{math}

Since \begin{math} \xi_{i}\end{math} satisfies all the conditions, hence it is correct to assume that \begin{math} \xi_{i}\end{math} will be the maximum value among all the inequalities.\\

\begin{math}\xi_{i} = \max \left ( 1 + \eta_{k}^{i} - \eta_{y} ^{i} \right ), k\neq y\end{math}

\begin{math}\xi_{i} =   1 + \underset{k\neq y}{\max} \;\eta_{k}^{i} - \eta_{y} ^{i}  \;and\; \xi_{i}\geq 0\end{math}

Therefore \begin{math} \xi_{i}\end{math} is always a positive function and can be written as :

\begin{math}\xi_{i} =   \left [  1 + \underset{k\neq y}{\max} \;\eta_{k}^{i} - \eta_{y} ^{i}\;\right ]_{+}\end{math}

The above is same as \begin{math} l_{cs}\left ( y^{i},\eta ^{i} \right ) \end{math}.\\ 

Hence, \begin{math} l_{cs}\left ( y^{i},\eta ^{i} \right ) = \xi _{i} \end{math}\\

So the constrained formulation can now be written in an unconstrained formulation which is same as (P2). As \begin{math} \xi _{i} \end{math} can be replaced by \begin{math} l_{cs}\end{math}. Hence by this expression (P2) can be derived from (P1) and vice-versa.

Suppose \begin{math} \left \{ \textbf{W}^{0}, \left \{ \xi _{i}^{0} \right \} \right \} \end{math} is the optimum for P1 equation.

So \begin{math}\xi_{i}\end{math} satisfies the cosntraints.\\

For \begin{math} \left \{ \textbf{W}^{0}, \left \{ \xi _{i}^{0} \right \} \right \} \end{math} the sum \begin{math}  \end{math} is minimum. Such that 

\begin{math}\xi_{i}^{0} =  \max \left ( 1 + \left \langle \textbf{w}^{k}, x^{i} \right \rangle - \left \langle \textbf{w}^{y^{i}}, x^{i} \right \rangle\right )  \;and\; \xi_{i}\geq 0
\end{math}\\

So, 
\begin{math}\xi_{i}^{0} =   1 + \underset{k\neq y}{\max} \;\eta_{k}^{i} - \eta_{y} ^{i}  \;and\; \xi_{i}\geq 0\end{math}

Since \begin{math}\xi_{i}^{0} \geq 0\end{math}  \;so \; \begin{math} 1 + \underset{k\neq y}{\max} \;\eta_{k}^{i} - \eta_{y} ^{i} \geq 0 \end{math}

As stated \begin{math} l_{cs}\left ( y^{i},\eta ^{i} \right ) \end{math}.\\ 

Hence \begin{math} l_{cs} = \xi_{i}^{0}\end{math}, So the problem (P2) after plugging the value of \begin{math} l_{cs} \end{math} becomes similar to (P1).\\

(P2) is :\\\\
\begin{math}\left \{ \widehat{\textbf{W}} \right \} = \underset{\textbf{}}{arg} \; \underset{  \textbf{W}}{min} \sum_{k=1}^{K} \left \| \textbf{w}^{k} \right \|^{2}_{2} + \sum_{i=1}^{n} l _{cs}\left ( y^{i}, \eta ^{i} \right )\end{math}

Then, Put value of \begin{math} l_{cs}\end{math}. The problem becomes,

\begin{math}\left \{ \widehat{\textbf{W}} \right \} = \underset{\textbf{}}{arg} \; \underset{  \textbf{W}}{min} \sum_{k=1}^{K} \left \| \textbf{w}^{k} \right \|^{2}_{2} + \sum_{i=1}^{n} \xi_{i}^{0}\end{math}

This problem is already solved as we know, \begin{math}\textbf{W}^{0}\end{math} minimizes  \begin{math}\sum_{k=1}^{K} \left \| \textbf{w}^{k} \right \|^{2}_{2}\end{math} and the \begin{math}\sum_{i=1}^{n} \xi_{i}^{0}\end{math} is just a constant which vanishes while minimizing.

So \begin{math}\textbf{W}^{0}\end{math} is also a solution for (P2).



 Lorem ipsum dolor sit amet, consectetur adipiscing elit. Duis feugiat vehicula dolor, sed ultricies leo. Phasellus euismod dictum felis in euismod. Proin pretium vel neque in placerat. Proin imperdiet egestas vulputate. Etiam faucibus accumsan ante non viverra. Duis ultrices ac odio vel sodales. In maximus gravida dolor, ut commodo lacus. Pellentesque ante massa, venenatis id aliquam et, posuere sed dui. Duis dignissim justo sit amet augue posuere fringilla. Suspendisse at nisi gravida, mattis justo sit amet, elementum elit. Praesent et massa ornare, consequat dui eget, ornare risus. Duis est nibh, sollicitudin nec mattis non, mattis in leo. Donec finibus justo sed massa sagittis, non fermentum nibh dictum. Pellentesque et congue purus. Donec porta pretium porttitor.

Morbi euismod risus eu tortor ornare malesuada. Nunc sed sollicitudin neque, efficitur rhoncus tellus. Cras malesuada augue arcu. Sed sem odio, tincidunt quis laoreet ac, facilisis ut nibh. Quisque gravida dolor at egestas aliquam. Aenean mollis massa sit amet enim mattis, vel fermentum tortor facilisis. Donec pellentesque est velit, vitae posuere lorem tristique ut. 
\end{mlsolution}

\begin{mlsolution}

\begin{math}f \left ( \textbf{w} \right ) = \sum_{i=1}^{n} \left [  1 - y^{i}<\textbf{w}, x^{i}> \right ]_{+}\end{math}\\

\begin{math}
= \sum_{i=1}^{n} \textbf{v}\left ( x^{i}, y^{i} \right ),\end{math} such that \begin{math}\textbf{v}\left ( x^{i}, y^{i} \right ) = \left\{\begin{matrix}
1 - y^{i}<\textbf{w}, x^{i}> &, y^{i}<\textbf{w}, x^{i}> \;<  1\\
0 & , y^{i}<\textbf{w}, x^{i}> \;\geq  1
\end{matrix}\right.
\end{math}

\begin{math}
f\left ( \textbf{w} \right )
\end{math} is just the sum of hinge losses, where hinge function is not differentiable at \begin{math}y^{i}<\textbf{w}, x^{i}> = 1\end{math} . To find the sub-differential of f at w, \\

\begin{math}
\frac{df\left ( \textbf{w} \right )}{d\textbf{w}} = \sum \frac{dv\left ( x^{i}, y^{i} \right )}{d\textbf{w}}
\end{math}\\

\begin{math}
\frac{dv\left ( x^{i}, y^{i} \right )}{d\textbf{w}} = \left\{\begin{matrix}
-y^{i}x^{i} &,  y^{i}<\textbf{w}, x^{i}> \;<  1\\ 
0 & , y^{i}<\textbf{w}, x^{i}> \;\geq  1
\end{matrix}\right.
\end{math}\\

\begin{math}
\bigtriangledown f = \frac{df\left ( \textbf{w} \right )}{d\textbf{w}}
\end{math}\\

Since \begin{math}\bigtriangledown f\end{math} is the subgradient of f. Therefore it should satisfy \\

\begin{math}
f\left ( \textbf{w}{}' \right ) \geq f\left ( \textbf{w}{} \right ) + \left \langle \bigtriangledown f, \textbf{w} - \textbf{w}{}' \right \rangle
\end{math}\\

As given in the question, 
g\begin{math} = \sum_{i=1}^{n}h_{i} \end{math} where  \\

\begin{math}
h_{i} = \left\{
\begin{matrix}
-y^{i}x^{i} &,  y^{i} \left \langle \textbf{w}, x^{i} \right \rangle \;<  1\\ 
0 & , y^{i} \left \langle \textbf{w}, x^{i} \right \rangle \;\geq  1
\end{matrix}\right.
\end{math}\\

So by the equations \textbf{g} appears to be same as \begin{math}f\left ( \textbf{w} \right ).\end{math} This can be proved by showing that for every \begin{math}\textbf{w}{}' \in \mathbb{R}^{d}, f\left ( \textbf{w}{}' \right ) \geq f\left ( \textbf{w} \right ) + \left \langle g, \textbf{w}{}' - \textbf{w} \right \rangle:\end{math}\\

\begin{math}
f\left ( \textbf{w}{}' \right ) = \sum_{i=1}^{n} \left [  1 - y^{i}<\textbf{w}{}', x^{i}> \right ]_{+} \end{math}\\

\begin{math}f\left ( \textbf{w} \right ) = \sum_{i=1}^{n} \left [  1 - y^{i}<\textbf{w}, x^{i}> \right ]_{+}
\end{math}\\

\begin{math}
f\left ( \textbf{w}{}' \right ) - f\left ( \textbf{w} \right ) = \sum_{i=1}^{n} \left [  1 - y^{i}<\textbf{w}{}', x^{i}> \right ]_{+} - \sum_{i=1}^{n} \left [  1 - y^{i}<\textbf{w}, x^{i}> \right ]_{+}
\end{math}\\

\begin{math}
f\left ( \textbf{w}{}' \right ) - f\left ( \textbf{w} \right ) = \sum_{i=1}^{n} \left [  1 - y^{i}<\textbf{w}{}', x^{i}> \right ]_{+} - \left [  1 - y^{i}<\textbf{w}, x^{i}> \right ]_{+}
\end{math}\\

Let's say that 
\begin{math}
f\left ( \textbf{w}{}' \right ) - f\left ( \textbf{w} \right ) = \sum_{i=1}^{n} s^{i}
\end{math}\\

So \begin{math}s^{i}\end{math} will be :\\

\begin{math}
s^{i} = \left\{\begin{matrix}
\left ( 1 - y^{i}<\textbf{w}{}', x^{i}>  \right ) - \left ( 1 - y^{i}<\textbf{w}, x^{i}> \right ) & ,y^{i}.<\textbf{w}{}', x^{i}> \; <  1&,  y^{i}.<\textbf{w}, x^{i}> \;< 1\\ 
- \left ( 1 - y^{i}<\textbf{w}, x^{i}> \right ) & ,y^{i}.<\textbf{w}{}', x^{i}> \; \geq  1 & ,  y^{i}.<\textbf{w}, x^{i}> \;< 1\\ 
\left ( 1 - y^{i}<\textbf{w}{}', x^{i}>  \right ) & ,y^{i}.<\textbf{w}{}', x^{i}> \; <  1 & ,  y^{i}.<\textbf{w}, x^{i}> \; \geq 1\\ 
0 & ,y^{i}.<\textbf{w}{}', x^{i}> \; \geq  1 & ,  y^{i}.<\textbf{w}, x^{i}> \; \geq 1
\end{matrix}\right.
\end{math}\\

The 1st part of \begin{math}s^{i}\end{math} can be simplified as following :

\begin{math}
=\left ( 1 - y^{i}<\textbf{w}{}', x^{i}>  \right ) - \left ( 1 - y^{i}<\textbf{w}, x^{i}> \right )\end{math}\\

\begin{math}= y^{i} \left (  <\textbf{w}{}', x^{i}> - <\textbf{w}, x^{i}>\right )\end{math}\\

\begin{math}= y^{i} \left (<\textbf{w} -\textbf{w}{}', x^{i}>\right )\end{math}\\

So \begin{math}s^{i}\end{math} will be :\\

\begin{math}
s^{i} = \left\{\begin{matrix}
y^{i} \left (<\textbf{w} -\textbf{w}{}', x^{i}>\right ) & ,y^{i}.<\textbf{w}{}', x^{i}> \; <  1&,  y^{i}.<\textbf{w}, x^{i}> \;< 1\\ 
- \left ( 1 - y^{i}<\textbf{w}, x^{i}> \right ) & ,y^{i}.<\textbf{w}{}', x^{i}> \; \geq  1 & ,  y^{i}.<\textbf{w}, x^{i}> \;< 1\\ 
\left ( 1 - y^{i}<\textbf{w}{}', x^{i}>  \right ) & ,y^{i}.<\textbf{w}{}', x^{i}> \; <  1 & ,  y^{i}.<\textbf{w}, x^{i}> \; \geq 1\\ 
0 & ,y^{i}.<\textbf{w}{}', x^{i}> \; \geq  1 & ,  y^{i}.<\textbf{w}, x^{i}> \; \geq 1
\end{matrix}\right.
\end{math}\\

\begin{math}
f\left ( \textbf{w}{}' \right ) \geq f\left ( \textbf{w} \right ) + \left \langle g, \textbf{w}{}' - \textbf{w} \right \rangle
\end{math}

We can also prove this if we satisy this : \\

\begin{math}
f\left ( \textbf{w}{}' \right ) - f\left ( \textbf{w} \right ) \geq  \left \langle g, \textbf{w}{}' - \textbf{w} \right \rangle
\end{math}

So the i-th element of the expression on the LHS is \begin{math}
s^{i}.\end{math}\\

Let's see RHS of the condition:\\

\begin{math}\left \langle g, \textbf{w}{}' - \textbf{w} \right \rangle = \sum_{i=1}^{n} h^{i}\left \langle \textbf{w}{}' - \textbf{w} \right \langle \\
\end{math}

\textbf{i}th element of \begin{math}
\left \langle g, \textbf{w}{}' - \textbf{w} \right \rangle, b^{i} = \left\{\begin{matrix}
\left \langle -y^{i}x^{i}, \textbf{w}{}' - \textbf{w} \right \rangle & ,y^{i}\left \langle \textbf{w}, x^{i} \right \rangle < 1\\ 
0 & ,y^{i}\left \langle \textbf{w}, x^{i} \right \rangle \geq 1
\end{matrix}\right.
\end{math}\\\\

So, simplifying and using the property of inner product\\\\

\begin{math}
b^{i} = \left\{\begin{matrix}
-y^{i} \left \langle \textbf{w}{}' - \textbf{w}, x^{i}\right \rangle & ,y^{i}\left \langle \textbf{w}, x^{i} \right \rangle < 1\\ 
0 & ,y^{i}\left \langle \textbf{w}, x^{i} \right \rangle \geq 1
\end{matrix}\right.
\end{math}\\

\begin{math}
b^{i} = \left\{\begin{matrix}
y^{i} \left \langle \textbf{w} - \textbf{w}{}', x^{i} \right \rangle & ,y^{i}\left \langle \textbf{w}, x^{i} \right \rangle < 1\\ 
0 & ,y^{i}\left \langle \textbf{w}, x^{i} \right \rangle \geq 1
\end{matrix}\right.
\end{math}\\

Now comparing LHS and RHS both. There are two cases, First when \begin{math}y^{i}.\left \langle \textbf{w}, x^{i} \right \rangle < 1 \end{math}\\

Then, \begin{math}b^{i}, s^{i}\end{math}\\

\begin{math}
s^{i} = \left\{\begin{matrix}
y^{i} \left \langle \textbf{w} -\textbf{w}{}', x^{i}\right \rangle & ,y^{i}.\left \langle \textbf{w}{}', x^{i} \right \rangle \; <  1\\ 
- \left ( 1 - y^{i}\left \langle \textbf{w}, x^{i}\right \rangle \right ) & ,y^{i}.\left \langle \textbf{w}{}', x^{i}\right \rangle \; \geq  1\\ 

\end{matrix}\right.
\end{math}\\

which is equal to \\

\begin{math}
s^{i} = \left\{\begin{matrix}
y^{i} \left \langle \textbf{w} -\textbf{w}{}', x^{i} \right \rangle & ,y^{i}.\left \langle \textbf{w}{}', x^{i}\right \rangle \; <  1\\ 
y^{i}\left \langle \textbf{w}, x^{i}\right \rangle -1 & ,y^{i}.\left \langle \textbf{w}{}', x^{i}\right \rangle \; \geq  1\\ 

\end{matrix}\right.
\end{math}\\

So when \begin{math}y^{i}.\left \langle \textbf{w}{}', x^{i} \right \rangle \; <  1\end{math} then \; \begin{math} s^{i} = b^{i} = y^{i} \left \langle \textbf{w} -\textbf{w}{}', x^{i} \right \rangle \end{math} \\

And when \begin{math}y^{i}.\left \langle \textbf{w}{}', x^{i} \right \rangle \; \geq  1\end{math}
 then \; \begin{math} s^{i} \geq b^{i}\end{math}. Which can be observed as follows :
 
Since \begin{math} y^{i}.\left \langle \textbf{w}{}', x^{i} \right \rangle \; \geq  1 \end{math} \\
Therefore, \begin{math} - y^{i}.\left \langle \textbf{w}{}', x^{i} \right \rangle \; \leq  -1 \end{math}\\

Adding \begin{math} y^{i}.\left \langle \textbf{w}, x^{i} \right \rangle \end{math} on both the sides of inequality,

Therefore, \begin{math} y^{i}.\left \langle \textbf{w} - \textbf{w}{}', x^{i} \right \rangle \; \leq  y^{i}.\left \langle \textbf{w}, x^{i} \right \rangle - 1 \end{math}

Which is same as \begin{math} b^{i} \leq s^{i} \end{math}

For the second case \begin{math}
y^{i}.\left \langle \textbf{w}, x^{i} \right \rangle \geq 1
\end{math}\\

Then, \begin{math}b^{i}, s^{i}\end{math}\\

\begin{math}
b^{i} = 0 ,
\end{math} and\\

\begin{math}
s^{i} = \left\{\begin{matrix}
1 - y^{i}\left \langle \textbf{w}{}', x^{i}\right \rangle & ,y^{i}.\left \langle \textbf{w}{}', x^{i} \right \rangle \; <  1\\ 
0 & ,y^{i}.\left \langle \textbf{w}{}', x^{i}\right \rangle \; \geq  1\\ \end{matrix}\right.
\end{math}\\ 

So when \begin{math}y^{i}.\left \langle \textbf{w}{}', x^{i} \right \rangle \; \geq  1\end{math} then \; \begin{math} s^{i} = b^{i} = 0 \end{math} \\

and when \begin{math}y^{i}.\left \langle \textbf{w}{}', x^{i} \right \rangle \; <  1\end{math}
then \; \begin{math} s^{i} > b^{i}\end{math}. Which can be observed as follows :
 
So, \begin{math}y^{i}.\left \langle \textbf{w}{}', x^{i} \right \rangle \; <  1 \end{math} \\

So, \begin{math} 1- y^{i}.\left \langle \textbf{w}{}', x^{i} \right \rangle \; >  0 \end{math} \\

Since \begin{math} s^{i} = 1- y^{i}.\left \langle \textbf{w}{}', x^{i} \right \rangle \; \end{math}and \begin{math} b^{i} = 0 \end{math} \\

so the inequality becomes, \begin{math} s^{i} > b^{i} \end{math}\\

Hence \begin{math} s^{i} \geq b^{i} \end{math}\\

So concluding from both the cases, we can say that \begin{math} s^{i} \geq b^{i} \end{math} for any \textbf{i}\\

Therefore, \begin{math}
f\left ( \textbf{w}{}' \right ) - f\left ( \textbf{w} \right ) \geq  \left \langle g, \textbf{w}{}' - \textbf{w} \right \rangle
\end{math}


\begin{math}
f\left ( \textbf{w}{}' \right ) \geq f\left ( \textbf{w} \right ) + \left \langle g, \textbf{w}{}' - \textbf{w} \right \rangle
\end{math}

Hence, \textbf{g} is a member of the subdifferential of f at \textbf{w}.
\end{mlsolution}


\begin{mlsolution}
Part 1 solution:\\\\
Used k-NN algorithm with the Euclidean metric to perform classification for different values of k = 1, 2, 3, 5, 10.
\\The test errors (no. of the 20K points that were incorrectly classified) :\\\\
For k = 1, No. of points incorrectly classified = 4815\\ \\
\begin{math} Error = \frac{4815}{20000} =  0.24075 \end{math}\\ \\
For k = 2, No. of points incorrectly classified = 4815 \\ \\
\begin{math} Error = \frac{4815}{20000} =  0.24075 \end{math} \\ \\
For k = 3, No. of points incorrectly classified = 3872 \\ \\ 
\begin{math} Error = \frac{3872}{20000} =  0.1936 \end{math} \\ \\
For k = 5, No. of points incorrectly classified = 3581 \\ \\
\begin{math} Error = \frac{3581}{20000} =  0.17905 \end{math} \\ \\
For k = 10, No. of points incorrectly classified = 3348 \\ \\ 
\begin{math} Error = \frac{3348}{20000} =  0.16915 \end{math} \\ \\
Figure \textbf{4}, Graph
showing test accuracies (fraction of the 20K points that were correctly classified) vs k.\\

\begin{figure}[th]%
\centering
\includegraphics[width=0.6\columnwidth]{Q1-6-part1.png}%
\caption{Plot of Test accuracies vs k }%
\label{fig:accuracy}%
\end{figure}


\textbf{Observation:} As the value of k increases the accuracy of our experiment increases. So we can infer that on increasing the no. of nearest neighbours the accuracy of k-NN may increase, but this we have observed only for small values of k. \\\\

\begin{center}
 \begin{tabular}{|c|c|} 
 \hline
 \textbf{k} & Accuracy \\ [0.5ex] 
 
 1 & 0.75925 \\ 
 
 2 & 0.75925   \\
 
 3 & 0.8064   \\
 
 5 & 0.82095   \\
 
 10 & 0.8326   \\ [1ex] 
 \hline
\end{tabular}
\end{center}

\textbf{Part 2 solution:}\\

The validation technique used : k-fold validation, the given training data is partitioned into k equal sized subsamples. Of the k subsamples, a single subsample is the validation data for testing the model (or our new test data), and remaining k − 1 subsamples make up our training data. Repeat this process k times (the folds), with each of the k subsamples used exactly once as the validation data. Take the average of the accuracies of the k results obtained from the folds to give an accuracy.

Used 5-fold validation, with the following possible values of k in k-NN = [1, 2, 3, 5, 10, 20]
Results are as follows : 
\\ k = 2, 5-fold accuracies = \begin{math} \left ( \; 0.7628, \; 0.76475, \; 0.75867, \; 0.7653, \; 0.7578 \right ) \end{math}
\\ k = 3, 5-fold accuracies = \begin{math} \left ( \; 0.80825, \; 0.81108, \; 0.80375, \; 0.8064, \; 0.801 \right ) \end{math}
\\ k = 5, 5-fold accuracies = \begin{math} \left ( \; 0.81816, \; 0.825, \; 0.8154, \; 0.8179, \; 0.81558 \right ) \end{math}
\\ k = 10, 5-fold accuracies = \begin{math} \left ( \; 0.8274, \;0.8346, \;0.8254, \;0.82875, \;0.8285 \right ) \end{math}
\\ k = 1, 5-fold accuracies = \begin{math} \left ( \; 0.7628, \;0.76475, \;0.7586, \;0.7653, \;0.7578 \right ) \end{math}
\\ k = 20, 5-fold accuracies = \begin{math} \left ( \; 0.8289, \;0.83558, \;0.82816, \;0.8319, \;0.83075 \right ) \end{math}

\begin{center}
 \begin{tabular}{|c|c|} 
 \hline
 k & avg(Accuracy) \\ [0.5ex] 
 2 & 0.7618 \\ 
 3 & 0.8061   \\
 5 & 0.8184   \\
 10 & 0.8289   \\
 1 & 0.7618   \\
 20 & 0.83106   \\ [1ex] 
 \hline
\end{tabular}
\end{center}

So a good value of k, based on our validation technique is the one with highest accuracy which is obtained in the case when k = 20, so k can be chosen as 10 or 20 as not much difference.

\newpage

\textbf{Part 3 solution :}\\

I tried this for different k values as I faced an issue that my training process for LMNN metric got killed when training from whole data. So instead of training for whole data, learned the LMNN metric using a fraction of data. So the results obtained is:\\\\For k = 10, and 1/6th of the training data LMNN metric is learnt and then accuracy is calculated using this metric instead of euclidean distance.\\\\Accuracy obtained is 0.83625 for k = 10 and 1/6th of training data.

\newpage

\textbf{Extra Credit part :}\\

Using the ITML technique with 1/4th of the training data and k = 10 and fixing the no. of constraints to 50, the accuracy obtained is = 0.8273.\\Using the ITML technique with 1/4th of the training data and k = 10 and fixing the no. of constraints to 100, the accuracy obtained is = 0.8298.\\\\Thus storing and using the second observation.
\\\\Also using the LSML technique with 1/4th of the training data and k = 10 and fixing the no. of constraints to 100, the accuracy obtained is =  0.81415.

\end{mlsolution}
\end{document}\\
